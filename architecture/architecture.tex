% Created 2013-11-04 Mon 23:36
\documentclass[11pt]{article}
\usepackage[utf8]{inputenc}
\usepackage[T1]{fontenc}
\usepackage{graphicx}
\usepackage{longtable}
\usepackage{hyperref}
\usepackage{svn}
\usepackage[T1]{fontenc}
\usepackage{mathpazo}
\usepackage[margin=1.3in]{geometry}
\linespread{1.05}
\usepackage[scaled]{helvet}
\usepackage{courier}
\usepackage{varioref}
\usepackage[usenames,dvipsnames]{color}
\usepackage{hyperref}
\hypersetup{colorlinks=true,linkcolor=blue,urlcolor=RawSienna}
\floatplacement{figure}{H}
\floatplacement{table}{H}
\newcommand{\hilight}[1]{\colorbox{yellow}{#1}}

\title{Solution Architecture Definition for ``Mirroring of Virtual-labs at IITD''}
\author{Suraj Samal}
\date{04 November 2013}

\begin{document}

\maketitle

\setcounter{tocdepth}{3}
\tableofcontents
\vspace*{1cm}

\listoftables
\listoffigures


\$\^{}{1}\$ FOOTNOTE DEFINITION NOT FOUND: 1

\section{Basic Architecture}
\label{sec-1}

\subsection{Overview}
\label{sec-1.1}

   Below is an overview of the overall system
   describing all the actors, entities and their interfaces:
   \hyperref[sec-1.1]{ ./overview.jpg }
\subsection{Actors}
\label{sec-1.2}

\subsubsection{Lab Developer}
\label{sec-1.2.1}

   An person who has agreed to use the services of VLEAD as per the
   \textbf{terms of association} and follows certain standard processes to
   maintain his/her lab during its development life-cycle. In
   specific, the roles are as follows:
\begin{itemize}
\item Checkin the lab contents (sources,dependencies, scripts and other files) into a lab-depository.
\item Keep updating the lab-depository with newer revisons of lab contents.
\item Instantiate a test lab-instance for testing and debugging issues.
\item Instantiate a live lab-instance.
\item View live lab-instance statistics.
\end{itemize}
\subsubsection{Lab Administrator}
\label{sec-1.2.2}

   An actor who is responsible for administering all the hosted
   labs. In specific, the roles are as follows:
\begin{itemize}
\item Allocate a unique labid and a depository(collection of
       repositories) to a lab
\item Allocation of resources(physical machines,ip address pools,
       vmid pools) to the labmanager and vmmanager
\end{itemize}
\subsubsection{Lab User}
\label{sec-1.2.3}

   These are end-users who use the virtual-labs and its experiments
\subsection{Entities}
\label{sec-1.3}

\subsubsection{LabDepository}
\label{sec-1.3.1}


     All labs are allocated a unique-id and a lab-depository by the
     labs administrator. A lab-depository represents a collection of
     various repositories associated with a lab.

\begin{description}
\item [lab-depository -] An \textbf{Object} describing the property of all
                    repositories of a particular lab

\begin{description}
\item [labid -] Unique identifier of the lab
\item [labinfo -] \textbf{Object} describing basic properties of a lab

\begin{description}
\item [labinst -] One of the defined \textbf{enumerations} ( IITB, IITK, IIITH ,,,)
\item [labdisc -] One of the defined \textbf{enumerations} ( chemical, mechanical \ldots{}..)
\item [labos -] \textbf{Object} describing a particular operating-system version

\begin{description}
\item [osname -] Name of the operating system
\item [osversion -] Specific version of the operating system
\end{description}

\end{description}

\item [labmetadata -] A structured \textbf{object} representation of depository
                     contents describing the number of repos present,
                     actual repos present, their type . This is regenerated
                     everytime the lab-developer makes a commit to a
                     repository.

\begin{description}
\item [numrepos -] Sum of all repositories present in the repository
\item [repo -] A repository \textbf{object} which refers to a svn, git or bzr repository

\begin{description}
\item [repoid -] Identification text that can be used to checkout the repository. (Eg: cse01, mech09 )
\item [reponame -] Display text (Eg: Frontend, Backend, UI etc)
\item [repotype -] One of the supported \textbf{enumerated} types - (git, svn, bzr)
\item [revsnum -] Number of revisions of the repository ( Eg: 20 )
\item [rev -] \textbf{Object} defining a particular repository revision

\begin{description}
\item [revno -] Unique revision number generated by the repository tool. ( Eg: 10 )
\item [date -] Date/Time the revision was checked into the repository. (Eg: 2013-11-10 16:30)
\item [user -] Text representing user who checked the revision. (Eg: ramakrishna)
\item [diskspace -] Approximate disk-space required. (Eg: 30G)
\item [ram -] Approximate memory required. (Eg: 256M)
\item [staticdeps -] An \textbf{object} describing a list of packages the lab depends on. (Eg: apache2, opencv)

\begin{description}
\item [dep1]
\item [dep2]
                      .
                      .
\item [depn]
\end{description}

\item [runtimedeps -] An \textbf{object} describing a list of services to be enabled/started. Services may mean
                                standard packages (eg. apache2) or other custom made scripts (Eg: backup)
                                to be configured during installation of the lab.

\begin{description}
\item [dep1]
\item [dep2]
                      .
                      .
                      .
\item [depn]
\end{description}

\item [size -] Number representing the size of the particular repository revision (\textbf{Optional})
\end{description}

\end{description}

\end{description}

\end{description}

\end{description}
\subsubsection{Lab}
\label{sec-1.3.2}


    An instance of a lab (inactive)  which refers to a complete set of
    properties that can be used to instantiate a particular lab
    revision. All these properties can be loaded directly from the
    lab-depository by using its unique labid, unique repoid and a
    unique revision no.

\begin{description}
\item [lab -]  \textbf{Object} describing an lab

\begin{description}
\item [labid -] Unique id to identify the lab from others
\item [labinfo -] \textbf{Object} describing basic properties of a lab
\item [repo -] \textbf{Object} describing a particular repository of a lab
\item [rev -] \textbf{Object} describing a particular revision of a particular
             repository of a lab
\end{description}

\end{description}
\subsubsection{LabManager}
\label{sec-1.3.3}


     An entity that monitors a set of hosts, accepts requests for
     creation, modification and deletion of labinstances and sends
     request to appropriate vm-manager for life-cycle management of
     labinstances

\begin{description}
\item [labmanager -] An entity responsible for managing the various vm-managers

\begin{description}
\item [labmanagerid -] Unique id to describe a labmanager
\item [hosts -] \textbf{Object} representation of a list of physical-hosts

\begin{description}
\item [host1 -] \textbf{Object} representation of a physical host (described later)
            .
            .
            .
\item [host2 -]
            .
            .
            .
\item [host3 -]
\end{description}

\item [runtime] runtime characterstics of the labmanager

\begin{description}
\item [start$_{\mathrm{time}}$ -] timestamp the labmanager was instantiated
\end{description}

\end{description}

\end{description}
\subsubsection{Host}
\label{sec-1.3.4}


     A physical host entity managed by a lab-manager and hosting a single vm-manager
\begin{description}
\item [Host -] Entity representing a physical host

\begin{description}
\item [hostname -] Common name of the host
\item [vmmgr -] \textbf{Object} representation of the vm-manager
                         (described later) managing the host
\item [hostid -] Unique-id representation of the host
\item [hostip -] IPaddress of the physical host
\item [resource -]  \textbf{Object} representation of resources of the physical host

\begin{description}
\item [diskspace -] (Eg. 2000GB)
\item [mem -] (Eg. 64GB)
\item [cpu -] (Eg. 2)
\end{description}

\item [runtime -] Runtime properties of the host

\begin{description}
\item [status -] one of running, stopped, shutoff
\item [start$_{\mathrm{time}}$ -] timestamp the host was started
\item [useddiskspace -] (Eg. 100GB)
\item [usedmem -] (Eg. 20GB)
\item [usedcpu -] (Eg. 1)
\end{description}

\end{description}

\end{description}
\subsubsection{VMManager}
\label{sec-1.3.5}


     An entity that is responsible for managing virtual machines(vms)
     on a particular host
\begin{description}
\item [vmmgr -] Entity describing an instance of a vm-manager
                   residing on a physical machine

\begin{description}
\item [vmmgrid -] Unique id to represent the vm-manager
\item [vms -] List of vm objects

\begin{description}
\item [vm1 -] \textbf{Object} representation of a vm (described later)
\item [vm2 -]
\end{description}

\item [vmn -]
\item [resources -] \textbf{Object} representation of resources

\begin{description}
\item [vmids -] List of available vmids

\begin{description}
\item [vmid1 -]
\item [vmid2 -]
                       .
\item [vmidn -]
\end{description}

\item [ips -] List of available ips

\begin{description}
\item [ip1 -]
\item [ip2 -]
                       .
                       .
\item [ipn -]
\end{description}

\end{description}

\item [runtime -] Runtime properties

\begin{description}
\item [status -] up, down, stopped
\item [start$_{\mathrm{time}}$ -] start timestamp
\end{description}

\end{description}

\end{description}
\subsubsection{VM}
\label{sec-1.3.6}


    A VM is a running instance of a lab.

\begin{description}
\item [vm -] An active instance of a lab that runs on a specified host

\begin{description}
\item [guid -] Global Universal id of the vm generated to identify the
\end{description}

\item [vmid -] Unique identification of a vm amoung its current running
      VMs. This is allocated from a defined pool of ids when the vm is
      created and re-sent to the pool when the vm gets destructed.
\item [vmname -] Common name to identify the VM instance.
\item [vmos -] Operating system \textbf{object} of the running vm.

\begin{description}
\item [osname -] Name of the operating system
\item [osversion -] Particular version of the operating system
\end{description}

\item [lab -] A particular instance of a lab associated with a vm
\item [runtime -] \textbf{Object} describing run-time properties of the vm

\begin{description}
\item [state -]  running, stopped, suspended, archived
\item [createddate -] Creation time-stamp of the VM
\item [modifieddate -] Modification time-stamp of the VM
\item [lastbackedup -] Timestamp when the vm was last backedup
\end{description}

\item [stats -] \textbf{Object} describing stats of a vm

\begin{description}
\item [userstats -] User-level statistics of the vm

\begin{description}
\item [userinfo -]
\end{description}

\item [perfstats -]

\begin{description}
\item [cpuinfo -]
\item [meminfo -]
\item [netinfo -]
\end{description}

\end{description}

\end{description}
\section{Relationships Model}
\label{sec-2}


\subsection{LabDepository - repository - revision}
\label{sec-2.1}


 [ Lab-Depository ] 1 -------------- *[ repo ] 1 ---------- * [ rev ]

\subsection{Lab - repository - revision}
\label{sec-2.2}


 [ Lab ] 1 -------- 1 [ repo ] 1 ------ 1 [ rev ]

\subsection{LabManager - host - vmmgr - vm - lab}
\label{sec-2.3}


 [ Labmanager ] * ------- * [ host ] 1 ------ 1 [ vmmgr ] 1 ------- * [ vm ] 1-------1 [ lab ]
\section{Workflows}
\label{sec-3}

\subsection{Lab Developer Workflows}
\label{sec-3.1}

\subsubsection{Create a Lab}
\label{sec-3.1.1}

\subsubsection{Update a Lab}
\label{sec-3.1.2}

\subsubsection{Test a Lab}
\label{sec-3.1.3}

\subsubsection{Release a Lab}
\label{sec-3.1.4}

\subsubsection{Delete a Lab}
\label{sec-3.1.5}

\subsubsection{Fetch Lab-Statistics}
\label{sec-3.1.6}

\subsection{Lab Administrator Workflows}
\label{sec-3.2}

\subsubsection{Create a Lab Repository}
\label{sec-3.2.1}

\subsubsection{Delete a Lab Repository}
\label{sec-3.2.2}

\subsubsection{Update Resource Information}
\label{sec-3.2.3}

\begin{itemize}
\item Physical Machine Resources
\item Network Parameters
\item VM Manager Information
\end{itemize}
\subsubsection{Update Lab Backup Schedule}
\label{sec-3.2.4}

\subsubsection{Take a Lab run-time snapshot}
\label{sec-3.2.5}

\subsubsection{Restore a Lab from its snapshot backup}
\label{sec-3.2.6}

\subsubsection{Deactivate a Lab}
\label{sec-3.2.7}

\subsubsection{Monitor VM Statistics}
\label{sec-3.2.8}

\subsubsection{Modify VM Run-time Parameters}
\label{sec-3.2.9}

\subsubsection{Purge a VM}
\label{sec-3.2.10}

\subsubsection{Purge VM logs}
\label{sec-3.2.11}

\subsection{User Workflows}
\label{sec-3.3}

\subsubsection{View a Lab}
\label{sec-3.3.1}

\subsection{Other Implicit Workflows}
\label{sec-3.4}

\subsubsection{Log Lab Information}
\label{sec-3.4.1}

\subsubsection{AutoPurge Lab History}
\label{sec-3.4.2}


\section{Component Model}
\label{sec-4}

\section{Network Architecture}
\label{sec-5}

  Presented below is a network architecture diagram of the proposed
  solution:
   \href{file://../network-infrastructure.jpg }{Network}
\section{Security Architecture}
\label{sec-6}

\section{Performance Model}
\label{sec-7}

\section{Reliability and Availability Model}
\label{sec-8}

\section{Backup Model}
\label{sec-9}

\section{Scalablility}
\label{sec-10}


\end{document}